\documentclass{article}

\usepackage[utf8]{inputenc}
\usepackage[british]{babel}
\usepackage[sc]{mathpazo}
\usepackage[T1]{fontenc}


% Or whatever. Note that the encoding and the font should match. If T1
% does not look nice, try deleting the line with the fontenc.
% \usepackage{lmodern} %optional

\usepackage{parskip,amsmath,array,booktabs,amssymb,color,alltt,bm,colortbl,bigints,wasysym,a4wide} 

\usepackage{mathtools}
\addtolength{\topmargin}{-15mm}

\newcommand{\iid}{\mbox{\,$\perp\!\!\!\perp$\,}}
\DeclareMathOperator{\logit}{logit}
\DeclareMathOperator{\E}{\mathbb{E}}
\DeclareMathOperator{\V}{\mathbb{V}}
\DeclareMathOperator{\C}{\mathbb{C}}
\DeclareMathOperator{\argmax}{\arg\max}
\DeclareMathOperator{\tr}{tr}
\DeclareMathOperator{\R}{\bfseries\textsf{R}}

\newcommand{\mc}[3]{\multicolumn{#1}{#2}{#3}}  
\newcommand{\bx}{\ensuremath{\bm{x}}}
\newcommand{\bbx}{\ensuremath{\bar{\bx}}}
\newcommand{\tbx}{\ensuremath{\tilde{\bx}}}
\newcommand{\bu}{\ensuremath{\bm{u}}}
\newcommand{\bv}{\ensuremath{\bm{v}}}
\newcommand{\bmf}{\ensuremath{\bm{f}}}
\newcommand{\bep}{\ensuremath{\bm{\varepsilon}}}
\newcommand{\by}{\ensuremath{\bm{y}}}
\newcommand{\bym}{\ensuremath{{\bm{y}_\text{mis}}}}
\newcommand{\byo}{\ensuremath{{\bm{y}_\text{obs}}}}
\newcommand{\bz}{\ensuremath{\bm{z}}}
\newcommand{\bmu}{\ensuremath{\bm{\mu}}}
\newcommand{\Sm}{\ensuremath{\Sigma}}
\newcommand{\Lm}{\ensuremath{\Lambda}}
\newcommand{\lm}[1]{\ensuremath{\lambda_{#1}}}
\newcommand{\Lmt}{\ensuremath{\Lm\Lm^\top}}
\newcommand{\diff}[1]{\ensuremath{{\frac{\partial}{\partial#1}}}}
\newcommand{\difff}[2]{\ensuremath{{\frac{\partial^2}{\partial#1\partial#2}}}}
\newcommand{\N}{\ensuremath{\mathcal{N}}}
\newcommand{\be}[1]{\ensuremath{{\bm{e}_{(#1)}}}}
\newcommand{\bt}{\ensuremath{{\bm{\theta}}}}
\newcommand{\bti}[1]{\ensuremath{{\bm{\theta}^{(#1)}}}}
\newcommand{\setfont}[1]{\fontsize{#1pt}{#1pt}\selectfont}
\newcommand{\code}[1]{{\texttt{#1}}}


\begin{document}

\title{Topics in Statistical Sciences 1 -- Exam exercise 2}
\author{S{\o}ren H{\o}jsgaard and Torben Tvedebrink}

\maketitle


This exercise is about Factor Analysis as discussed in three lectures
of Topics in Statistical Sciences 1. During the oral exam you will
have 20 min to present the exercise. You decide what topics to cover
and how to present them, however, we will ask questions to any part of
the exercise and presentation.

\section{Some theoretical results for Factor Analysis}
\label{sec:proof-non-decreasing}

\begin{enumerate}
\item Show that the Bartlett (1938) method for estimating the specific
  factor, $\tilde{\bmf}_i$ for a given observation, $\bx_i$, using
  weighted least squares is given by
  \begin{displaymath}
    \tilde{\bmf}_i = (\Lm^\top\Psi^{-1}\Lm)^{-1}\Lm^\top\Psi^{-1}(\bx_i-\bmu).
  \end{displaymath}

  In \textsl{R} these are obtained by setting \texttt{scores =
    "Bartlett"} in \texttt{factanal}.
\item For \texttt{scores = "regression"} we use the method of Thomson
  (1951). Let $\hat{f} = A(\bx-\bmu)$ and show that the expression,
  which minimises $\E[\|\hat{\bmf}-\bmf\|^2]$ is given by
  \begin{align}
    \hat{\bmf} &=  \Lm^\top(\Lmt+\Psi)^{-1}(\bx - \bmu)\notag\\
    &= (I_k + \Lm^\top\Psi^{-1}\Lm)^{-1}\Lm^\top\Psi^{-1}(\bx - \bmu)\label{eq:thomson}
  \end{align}
  For the last expression in \eqref{eq:thomson} you will make use of
  Woodbury's identity and the fact that for a regular matrix $J$:
  \begin{displaymath}
    J = I + J - I 
    \quad\text{from which we have}\quad
    J(I+J)^{-1} = I - (I+J)^{-1}
  \end{displaymath}
  Note that the Thomson approach coincides with $\E[\bmf\mid\bx]$,
  i.e.\ $\bmf$ is regressed on $\bx$.
\end{enumerate}


	** 1.2 **
	For scores = "regression" bruger vi metoden Thomson (1951). Lad $\tilde{f} = A(x-\mu)$ og vis, at det udtryk der minimer $\mathbb{E}[||\tilde{f}-f||^2]$ er givet ved 
	
	\begin{center}
		$\tilde{f} = \Lambda^T(\Lambda\Lambda^T+\Psi)^{-1}(x-\mu)$
	\end{center}
	
	** Besvarelse **
	
	Starter med at udregne udtrykket for $\mathbb{E}[||\tilde{f}-f||^2]$
	
	\begin{center}
		$\mathbb{E}[||\tilde{f}-f||^2]=\mathbb{E}[(\tilde{f}-f)^T(\tilde{f}-f)]$
		
		$=\mathbb{E}[(Ay-f)^T(Ay-f)]=\mathbb{E}[\text{tr}\lbrace(Ay-f)^T(Ay-f)\rbrace]$
		
		$=\text{tr}\mathbb{E}[(y^TA^TAy]+\mathbb{E}[f^Tf]-2\mathbb{E}[y^TA^Tf]$
		
		$=\mathbb{E}[\text{tr}(A^TAyy^T)]+\text{tr}(I_k)-2\text{tr}(\mathbb{E}[A^Tfy^T])$
		
		$=\text{tr}(A^T A \mathbb{E}[yy^T] ) + k-2\text{tr}\lbrace A^T\mathbb{E}[f(\Lambda f+\varepsilon)^T] \rbrace$
		
		$=\text{tr}(A^T A \mathbb{E}[(x-\mu)(x-\mu)^T] ) + k-2\text{tr}\lbrace A^T\mathbb{E}[ff^T)\Lambda^T] \rbrace$
		
		$=\text{tr}\lbrace A^TA\Sigma\rbrace +k-2\text{tr}\lbrace A^T\Lambda^T\rbrace$
	\end{center}
	
	Dernæst tages den afledte af ovenstående resultat og sættes lig 0
	
	\begin{center}
		$\frac{\partial}{\partial A}\mathbb{E}[||\tilde{f}-f||^2]=0$
	\end{center}
	
	Den afledte af $\mathbb{E}[||\tilde{f}-f||^2]$ mht. A er givet ved
	
	\begin{center}
		$A\Sigma+(A^T)^T\Sigma^T-2\Lambda^2=0$
	\end{center}
	
	A isoleres i udtrykket
	
	\begin{center}
		$2A\Sigma=2\Lambda^T\Rightarrow A=\Lambda^T\Sigma^{-1}=\Lambda^T(\Lambda\Lambda^T+\Psi)^{-1}$
	\end{center}
	
	Ovenstående udtryk for A indsættes i det givne udtryk for $\tilde{f}$
	
	\begin{center}
		$\tilde{f}=A(x-\mu)=\Lambda^T(\Lambda\Lambda^T+\Psi)^{-1}(x-\mu)$
	\end{center}
	
	Dermed er resultatet bevist.


In the discussion of the EM algorithm in the lecture, the result in
the M-step for $\Psi^{(t+1)}$ relied on the most recent estimate of
$\Lm^{(t+1)}$ (not $\Lm^{(t)}$ as would be the usual case in an EM
algorithm). However, this result relied implicitly on a generalised
version of the EM algorithm. Here we show a more directed result using
an ordinary EM algorithm.

\begin{enumerate}
  \setcounter{enumi}{2}
\item Show that the derivative of the expected log likelihood with
  respect to $\Psi^{-1}$ is given as
\begin{multline}
  \diff{\Psi^{-1}}q(\bt; \bti{t}) = \frac{n}{2}\Psi -
  \frac{n}{2}\E_{\bti{t}}[S\mid \bx] = \\
 ~\quad\frac{n}{2}\Psi -\frac{1}{2}\sum_{i=1}^n
 \{\bx_i\bx_i^\top - \Lm\E_{\bti{t}}[\bmf_i\mid\bx_i]\bx_i^\top -
  \bx_i\E_{\bti{t}}[\bmf_i\mid\bx_i]^\top\Lm^\top +
  \Lm(\E_{\bti{t}}[\bmf_i\bmf_i^\top\mid\bx_i])\Lm^\top\} \label{eq:psi}
\end{multline}
\item Use the result in \eqref{eq:psi} to derive the M-step for the EM
  algorithm, where we note that only the diagonal elements are needed.
\end{enumerate}

\section{Analysis of \texttt{carcass\underline{~}fa} data}

In the \texttt{gRbase} package (H{\o}jsgaard, 2005) the dataset
\texttt{carcassall} is available. A dataset
(\texttt{carcass\underline{~}fa}) containing a subset of the variables
is available on Moodle. From \texttt{?carcassall} we get the short
description of the dataset: ``Measurement of lean meat percentage of
344 pig carcasses together with auxiliary information collected at
three Danish slaughter houses''

The \texttt{carcass\underline{~}fa} dataset on Moodle contains measurements of 14
numeric measurements taken in order to assess the lean meat content:
{\sl
\begin{description}
\item[weight]
Weight of carcass

\item[lengthc]
Length of carcass from back toe to head (when the carcass hangs in the back legs)

\item[lengthf]
Length of carcass from back toe to front leg (that is, to the shoulder)

\item[lengthp]
Length of carcass from back toe to the pelvic bone

\item[Fat02, Fat03, Fat11, Fat12, Fat13, Fat14, Fat16] Thickness of
  fat layer at different locations on the back of the carcass (FatXX
  refers to thickness at (or rather next to) rib no). XX. Notice that
  02 is closest to the head

\item[Meat11, Meat12, Meat13]
Thickness of meat layer at different locations on the back of the carcass, see description above
\end{description}
}

Analyse the \texttt{carcass\underline{~}fa} data using factor
analysis. In your analysis you should assess the number of appropriate
factors, some visual output (graphics) to support the analysis and
make an attempt to interpret the analysis.


\end{document}

%%% Local Variables: 
%%% mode: latex
%%% TeX-master: t
%%% End: 
